\section{Video Synthesis}
\label{sec:video_synthesis}

Video synthesis is a complex task. One can think of video generation as a sequence of image generation tasks. Formally, a video is a sequence of images (or frames) that are shown in fast fashion, usually 24 frames per second (at the minimum, to get smooth video). Therefor to create a video of 5 seconds, you'll need 120 frames or images at the minimum. Additional complexity is the addition of the \textbf{time dimension}, which is not present in image generation tasks. The video should be \textbf{coherent} in time, meaning that the frames should be related to each other and should follow a logical sequence. Objects should not appear out of nowhere, there should be \textbf{smooth transition of motion} and correct \textbf{spatial relationships} between objects. We also have to deal with limited hardware resources, since video generation is extremely computationally expensive.

% There are a lot of techniques and models for video generation, and they can be divided into multiple categories:

% \begin{itemize}
%     \item \textbf{Frame prediction}: ...
%     \item \textbf{GAN based models}: such in the case of \cite{chu2020learning} the authors propose 
% \end{itemize}

Video synthesis can be based on four main models and methods:

\begin{enumerate}
    \item \textbf{Diffusion}: diffusion models for video generation extend the iterative refinement process used in image-based diffusion models, such as LDMs, to handle temporal dynamics required for videos. The key idea is to generate not just a sequence of independent images, but a continuous video where both the spatial content of each frame and the temporal coherence between frames are learned and generated simultaneously.
    \item \textbf{Spatial autoregressive}: spatial autoregressive models for video generation, as described in \cite{graves2013generating}, synthesize video by sequentially generating content in a patch-based fashion, where each patch in a frame is conditioned on previously generated patches. This method builds video frame-by-frame, ensuring spatial and temporal coherence across the entire sequence.
    \item \textbf{GAN}: similar to diffusion models, GANs can also be leveraged for video synthesis. 
    \item \textbf{Mask modeling}: mask modeling in video generation uses the technique of selectively masking parts of video frames during training to enhance the model's ability to learn spatial and temporal dependencies. By hiding portions of the video frames, the model is tasked with predicting and reconstructing the missing parts, forcing it to better understand the underlying structure and motion in the video.
\end{enumerate}

In the realm of video synthesis there are two paradigms: \textbf{divide and conquer} and \textbf{temporal autoregressive}.

\subsection*{Divide and Conquer}

Divide and conquer paradigm breaks the task of video synthesis into smaller, more manageable tasks. This approach can further by divided into 3 methods of approach:

\begin{enumerate}
    \item \textbf{Hierarchical architecture} for frame generation: in this approach, the process is split into \textbf{keyframe generation} and \textbf{frame filling}. Keyframes represent the critical moments of the video that establish the narrative, while the frame-filling models ensure smooth transitions between them. Global models focus on generating the main storyline through keyframes, and local models handle finer details, filling the gaps between the keyframes to maintain temporal consistency.
    \item \textbf{Multi-stage approach for video super-resolution}: \cite{brooks2022generating} suggested generating video by stages, first by creating low-resolution sequences with GAN and then applying super-resolution models to enhance them with StyleGAN3 \cite{stylegan3}.
    \item \textbf{Integrated keyframe and frame filling with mask modeling} simplifies long video generation by combining keyframe creation and frame filling into a unified process. Mask modeling hides specific parts of the video during training, keyframes and intermediate frames simultaneously, like done in CogVideo \cite{cogvideo}, which simplified mask modeling and combines these models into a single model.
\end{enumerate}

\subsection*{Temporal autoregressive}

Temporal autoregressive paradigm is an iterative approach where each frame is generated based on the previous one. The model is trained on sequences of video data, and the output from one timestep serves as the input for the next (the next frame is conditioned on the previous frame, which in theory ensures that the video is temporally coherent).

A significant amount of research has been conducted on temporal autoregressive models, leading to various improvement strategies and enhancements for this framework. Some of the most notable include:

\begin{itemize}
    \item \textbf{Latent space compression}: this method focuses on compressing videos to latent space to optimize computational needs. For example in \cite{zeng2024make} and \cite{gu2023reuse} they explored condensing video data into 3D latent. Compression aims to preserve essential features across dimensions to improve computational efficiency.
    \item \textbf{Incorporating temporal layers} to refine individual video clips, advancements were made, such as integrating \textbf{temporal layers} (such as attention and convolutional layers) into diffusion models, like in VideoLDM \cite{video_ldm} and \cite{gu2023reuse}. These temporal layers help the model better understand the temporal dynamics of video data.
    \item \textbf{Dual-phase training \& reuse strategies}: dual-training is a training strategy where the model is first trained unconditionally, and then learn to conditionally generate video like in Projected Latent Video Diffusion Model (PVDM) \cite{pvdm}. Given the abundance of unlabeled video data on the web, it's easy to see why this process is attractive. Another method to boost the model's ability to replicate long video sequences is a \textbf{reuse strategy}, which iteratively adds and removes noise during training to simulate natural variability, like in \cite{gu2023reuse}.
\end{itemize}

\subsection*{Spatial autoregressive models}

Spatial autoregressive models are adept at processing tokenized sequence inputs (they include the transformer architecture) which enable the segmentation of video frames into patches. By integrating video data features with the autoregressive power of transformers, these models become more capable of capturing temporal dynamics. In this method, there are two main approaches:

\begin{itemize}
    \item \textbf{Frame tokenization}: like in NUWA-Infinity \cite{nuwa_infinity}, the researchers transform video frames into patches and combine them with spatial positional encodings for more efficient data handling. These autoregressive models, similar to LDMs, compress video data into latent space. Compression techniques like \textbf{Discrete Cosine Transform (DCT)} are used in Transframer \cite{transframer} and by using VQGAN, they convert the frames into latent tokens.
    \item \textbf{Scaling with attention mechanism}: architectural modifications were made by incorporating specialized blocks, such as attention mechanism blocks tailored for spatiotemporal data. For example, Transframer \cite{transframer} integrated temporal and spatial annotations (time-steps, camera viewport, etc.) through cross-attention.
\end{itemize}

\subsection*{Challenges in video synthesis}

Because video synthesis is still developing to this day and is a complex task, there are a lot of open questions to be asked:

\begin{itemize}
    \item How to generate long videos while maintaining high temporal cohesion?
    \item Would it be possible to generate videos in high resolution, with limited computational resources?
    \item How to train video generation models with abundant, unlabeled video data, such as YouTube? Are there any open datasets for video generation?
    \item How to generate videos from text, or other modalities? Is it possible to edit generated videos so it fits our artistic needs?
    \item What is considered a long video? Frame count (e.g., 512, 1024 frames)? Duration (e.g., 3, 5 minutes)?
    \item How we would trade off between resolution (high resolution) of generated video, duration (long), and spatial coherence (smooth motion and logical spatial relationships)? Can we achieve all three?
    \item How to deal with abrupt scene transitions? How to ensure smooth transitions between scenes?
    \item How to measure and evaluate temporal-spatial consistency between video generation models? How to tell if one clip is smoother and more coherent than the other?
\end{itemize}

In \cite{long_video_survey} they surveyed papers in the field of video generation and propose a definition for 'long' videos to exceed 10 seconds long, assuming frame rate of 10fps, or equivalently 100 frames.








\subsection{Previous works}

In VideoGAN \cite{video_gan} (2016) the researchers proposed to use of the large amounts of unlabeled video in order to learn scene dynamics and motion, and capture temporal signals. They proposed a model with \textbf{spatio-temporal convolutional} architecture, and they extend GANs to video generation. They successfully created a video of 1 second long, the model explicitly models the foreground separately from the background, which allows the network to learn which objects are in motion, and which don't. The methodology in this work is autoregressive frame prediction.

In the case of \cite{video_generation_from_text}, where the goal is to generate video from text, they proposed a hybrid model that uses a conditional variational autoencoder (VAE) and a generative adversarial network (GAN). The VAE model will output a 'gist' which is a sketch (intermediate step), how the video will take form (colors, object layout) and the GAN model will apply a set of image filter kernels based on the input text to get an encoded text-gist feature vector and the framework predicts future frames.

NUWA-XL \cite{yin2023nuwa} is a diffusion model that uses 3D-UNet for keyframe generation (by global diffusion model) and keyframe filling (by local diffusion model). They also built a new dataset called 'FlintstonesHD' for benchmarking long video generation. Their method reduced inference time when generating 1024 frames from 7.55 minutes to 26 seconds (\textbf{a 94\% reduction in inference time}) on the same hardware.

% TODO: Add more context
In \cite{ge2022long} they used 3D VQ-GAN with hierarchical transformer architecture to generate long videos (1024 frames or more). They used standard benchmark datasets: UCF-101, Sky Time-lapse and Taichi-HD.

% TODO: Add more context
VideoGPT \cite{videogpt} uses VQ-VAE to encode video frames into discrete latent representations by using 3D convolutions and self-attention. It uses the GPT \cite{gpt} architecture

