\section{Variational Autoencoder}
\label{sec:vae}

Variational Autoencoder (VAE) \cite{vae} is generative model is used to learn the underlying distribution of data and generate new samples (similar to the dataset). The model consists of 3 main components: an encoder, latent space (sometimes called 'code vectors' or 'bottleneck layer') and a decoder. The main idea behind VAE is to use the autoencoder model \cite{autoencoder} \cite{autoencoder2} to compress large dimensional vectors (in our case, images) into smaller, low dimension vectors that represent the underlying features hidden within the input data. These code vectors are then fed into a decoder network which reconstructs the image (i.e. high dimensional vector).


\begin{figure}[h]
    \centering
    \begin{tikzpicture}
        % Center point of encoder
        \coordinate (E_CENTER) at (1, 1);
        \coordinate (INPUT_TEXT) at (-0.25, 0.5);



        % Draw input vector
        \draw ($(INPUT_TEXT) + (-0.25, -0.5)$) rectangle ($(INPUT_TEXT) + (0.25, 1.5)$) node[pos=.5] {$x$};
        \draw[->] ($(E_CENTER) + (-1, 0)$) -- (E_CENTER);

        








        % Draw the encoder
        \node at ($(E_CENTER) + (1, 1.5)$) {Encoder};
        
        \coordinate (A) at ($(E_CENTER) + (0, -1)$);
        \coordinate (B) at ($(E_CENTER) + (2, -0.5)$);
        \coordinate (C) at ($(E_CENTER) + (2, 0.5)$);
        \coordinate (D) at ($(E_CENTER) + (0, 1)$);
        \draw[fill=gray!30] (A) -- (B) -- (C) -- (D) -- cycle;
        
        % Define the coordinates for the first set of circles (3 neurons)
        \coordinate (n1) at ($(E_CENTER) + (0.5, -0.5)$);
        \coordinate (n2) at ($(E_CENTER) + (0.5, 0)$);
        \coordinate (n3) at ($(E_CENTER) + (0.5, 0.5)$);
        
        % Define the coordinates for the second set of circles (2 neurons)
        \coordinate (m1) at ($(E_CENTER) + (1.5, 0.25)$);
        \coordinate (m2) at ($(E_CENTER) + (1.5, -0.25)$);
        
        % Draw the first set of circles
        \foreach \i in {n1, n2, n3} {
            \filldraw[fill=white] (\i) circle (6pt);
        }
        
        % Draw the second set of circles
        \foreach \i in {m1, m2} {
            \filldraw[fill=white] (\i) circle (6pt);
        }
        
        % Draw arrows from each circle in the first set to each circle in the second set
        \foreach \i in {n1, n2, n3} {
            \foreach \j in {m1, m2} {
                \draw[-] (\i) -- (\j);
            }
        }






        % Draw code vector

        % Arrow in
        \coordinate (Z_TEXT) at ($(E_CENTER) + (3, 0)$);
        \draw[->] ($(E_CENTER) + (2, 0)$) -- ($(Z_TEXT) + (-0.25, 0)$);

        \draw ($(Z_TEXT) + (-0.25, -0.5)$) rectangle ($(Z_TEXT) + (0.25, 0.5)$) node[pos=.5] {$z$};
        
        % Arrow out
        \coordinate (D_CENTER) at ($(E_CENTER) + (4, 0)$);
        \draw[->] ($(Z_TEXT) + (0.25, 0)$) -- (D_CENTER);











        % Draw decoder
        \node at ($(D_CENTER) + (1, 1.5)$) {Decoder};
        
        \coordinate (A2) at ($(D_CENTER) + (0, 0.5)$);
        \coordinate (B2) at ($(D_CENTER) + (2, 1)$);
        \coordinate (C2) at ($(D_CENTER) + (2, -1)$);
        \coordinate (D2) at ($(D_CENTER) + (0, -0.5)$);
        \draw[fill=gray!30] (A2) -- (B2) -- (C2) -- (D2) -- cycle;
        
        % Define the coordinates for the first set of circles (2 neurons)
        \coordinate (n2_1) at ($(D_CENTER) + (0.5, 0.25)$);
        \coordinate (n2_2) at ($(D_CENTER) + (0.5, -0.25)$);
        
        % Define the coordinates for the second set of circles (3 neurons)
        % \coordinate (m2_1) at (1.5, 0.5);
        % \coordinate (m2_2) at (1.5, 1);
        % \coordinate (m2_3) at (1.5, 1.5);
        \coordinate (m2_1) at ($(n2_1) + (1, 0.25)$);
        \coordinate (m2_2) at ($(n2_1) + (1, -0.25)$);
        \coordinate (m2_3) at ($(n2_1) + (1, -0.75)$);
        
        % Draw the first set of circles
        \foreach \i in {n2_1, n2_2} {
            \filldraw[fill=white] (\i) circle (6pt);
        }
        
        % Draw the second set of circles
        \foreach \i in {m2_1, m2_2, m2_3} {
            \filldraw[fill=white] (\i) circle (6pt);
        }
        
        % Draw arrows from each circle in the first set to each circle in the second set
        \foreach \i in {n2_1, n2_2} {
            \foreach \j in {m2_1, m2_2, m2_3} {
                \draw[-] (\i) -- (\j);
            }
        }


        \coordinate (D_END) at ($(D_CENTER) + (2, 0)$);




        % Draw output vector
        \coordinate (X_OUT_TEXT) at ($(m2_2) + (2, 0)$);
        \draw[->] (D_END) -- ($(X_OUT_TEXT) + (-0.25, 0)$);
        
        \draw ($(X_OUT_TEXT) + (-0.25, -1)$) rectangle ($(X_OUT_TEXT) + (0.25, 1)$) node[pos=.5] {$\hat{x}$};
        
        % \node at (X_OUT_TEXT) {$\hat{x}$};
        
    \end{tikzpicture}
    \caption{Autoencoder architecture.}
\end{figure}


More formally, the encoder network takes an input data point $x$ and maps it to a latent space representation $z$, which is compressed representation of $x$. The input is a vector, therefor an image must be flattened from 2D to 1D vector. This flattening will become an issue later in image generation, as this action removes important spatial information and hidden structures in the image. Because of this, modifications were made to the VAE model which allows the capture of spatial information by using a CNN (Convolutional Neural Network) \cite{cnn} layers \cite{vae_cnn_example}, max pooling layers, and more. After compression, the latent vector $z$ is then passed onto the decoder for reconstruction. 

The reconstruction is learned by a reconstruction loss function, usually mean squared error (MSE) loss function:

\begin{equation}
    \text{MSE} = \frac{1}{N} \sum_{i=1}^{N} (y_i - \hat{y}_i)^2
\label{eq:mse}
\end{equation}

The MSE loss is common in image generation models, because this loss is used to ensure that the generated images closely resemble the original input images by comparing the distance between pixels (input image, output image). This objective motivates the model to reconstruct the image from latent vector to resemble the original image. However, this loss is not used alone usually, but in combination with more complex loss functions, as we will see later in VAE.

The code vectors learned by autoencoders, however, are one-to-one map of the input and code vector (deterministic mapping from input to code vectors). The model doesn't capture any semantic relationships between the data (e.g the code vectors of images of cats are scattered throughout the entire latent space, whereas in VAE, they are clustered together). The latent space in autoencoders is irregular and discontinuous, meaning a small change in the latent vector can lead to large unpredictable changes in the output. This makes interpolation in the latent space difficult. Variational autonecoder solve this problem. VAE regularizes the latent space by enforcing a prior distribution. This regularization leads to a smooth and continuous latent space (see figure \ref{fig:ae_vs_vae}), which allows the model to interpolate between the latent space smoothly, thus creating similar new images with different variations. VAEs also provide an explicit model of the data distribution by maximizing a variational lower bound on the likelihood of the data. In other words, VAE is probabilistic model instead of discrete mapper (like the autoencoder model).

\begin{figure}[h]
    \centering
    \includegraphics[scale=0.5]{images/autoencoder-vs-variational-autoencoder-point-vs-distribution-768x409.png}
    \caption{Illustration of mapping an input image to code vector (left) and mapping an input image to a distribution (right) \cite{ae_vs_vae}.}
    \label{fig:ae_vs_vae}
\end{figure}

At the heart of VAE lies the concept of latent variables (Appendix \ref{appendix:latent_variables}). Latent variables are hidden, unobserved variables that the model infers from the observed data (dataset). Latent variable models, such as VAE, take indirect approach to describing a probability distribution $p(x)$ over multi-dimensional variable $x$. Instead of directly writing the expression for $p(x)$, they model a joint distribution $p(x|z)$ of the data $x$ and an unobserved hidden latent variable $z$.

The variational aspect of VAE refers to the use of variational inference (VI) (Appendix \ref{appendix:variational_inference}). VI is used to used to approximate complex posterior distributions: 

\begin{equation}
p(z|x) = \frac{p(x|z) \cdot p(z)}{\int p(x|z) \cdot p(z) dz}
\label{eq:posterior}
\end{equation}

by transforming the problem into optimization problem. The denominator in eq. \ref{eq:posterior} is intractable because it involves integrating over all possible values of $z$, and $z$ is often relatively high-dimensional and its infeasible to evaluate exactly. Which is why VI is used, which approximates the true posterior distribution $p(z|x)$ with simpler, tractable distribution $q_\phi (z|x)$, parameterized by $\phi$.

The original VAE model uses fully connected layers at the encoder and decoder networks. However, in the image synthesis field, CNN layers (appendix \ref{appendix:cnn}) are used instead which are computationally less expensive and better capture the spatial information.

To generate an image, we first sample a latent variable $z$ from prior distribution $p(z)$, which is typically standard normal distribution $\mathcal{N}(0, 1)$. Then $z$ is passed to the decoder an an image $x$ is generated from the conditional distribution $p_\theta (x|z)$. 

\subsection{The Reparameterization Trick}
To enable backpropagation through the sampling process, VAEs use the reparameterization trick. This trick involves expressing the sampled latent variables $z$ as a deterministic function of the encoder's output and some random noise. Without this technique, backpropagation would not be possible through the sampling operation. The reason is that sampling is a non-differentiable operation, because sampling from a distribution involves randomness that does not have a gradient, and the gradients cannot be computed with respect to the parameters of the encoder. To make the sampling operation differentiable and thus allow gradients to flow through the network, the reparameterization trick is used. 

Specifically, if $\mu$ and $\sigma$ are the mean and standard deviation vectors outputted by the encoder, we can write:

\begin{equation}
    z = \mu + \sigma \cdot \epsilon
\end{equation}

where $\epsilon \sim \mathcal{N}(0, 1)$ is a standard normal random variable. This $\epsilon$ will not change throughout the training reigime. It is sampled once and fixed in place. This trick allows us instead of having full stochastic node that blocks flow of gradients, to having two parts: one where we can do backpropagation, and another part which is still stochastic but which we don't want to train because its fixed.


\begin{figure}[h]
    \centering
    \begin{tikzpicture}
        % Center point of encoder
        \coordinate (E_CENTER) at (1, 1);
        \coordinate (INPUT_TEXT) at (-0.25, 0.5);



        % Draw input vector
        \draw ($(INPUT_TEXT) + (-0.25, -0.5)$) rectangle ($(INPUT_TEXT) + (0.25, 1.5)$) node[pos=.5] {$x$};
        \draw[->] ($(E_CENTER) + (-1, 0)$) -- (E_CENTER);

        








        % Draw the encoder
        \node at ($(E_CENTER) + (1, 1.5)$) {Encoder $q_\phi(z|x)$};
        
        \coordinate (A) at ($(E_CENTER) + (0, -1)$);
        \coordinate (B) at ($(E_CENTER) + (2, -0.5)$);
        \coordinate (C) at ($(E_CENTER) + (2, 0.5)$);
        \coordinate (D) at ($(E_CENTER) + (0, 1)$);
        \draw[fill=gray!30] (A) -- (B) -- (C) -- (D) -- cycle;
        
        % Define the coordinates for the first set of circles (3 neurons)
        \coordinate (n1) at ($(E_CENTER) + (0.5, -0.5)$);
        \coordinate (n2) at ($(E_CENTER) + (0.5, 0)$);
        \coordinate (n3) at ($(E_CENTER) + (0.5, 0.5)$);
        
        % Define the coordinates for the second set of circles (2 neurons)
        \coordinate (m1) at ($(E_CENTER) + (1.5, 0.25)$);
        \coordinate (m2) at ($(E_CENTER) + (1.5, -0.25)$);
        
        % Draw the first set of circles
        \foreach \i in {n1, n2, n3} {
            \filldraw[fill=white] (\i) circle (6pt);
        }
        
        % Draw the second set of circles
        \foreach \i in {m1, m2} {
            \filldraw[fill=white] (\i) circle (6pt);
        }
        
        % Draw arrows from each circle in the first set to each circle in the second set
        \foreach \i in {n1, n2, n3} {
            \foreach \j in {m1, m2} {
                \draw[-] (\i) -- (\j);
            }
        }






        % Draw middle
        \coordinate (Z_TEXT) at ($(E_CENTER) + (3, 0)$);

        \coordinate (MIDDLE_BEGIN) at ($(Z_TEXT) + (-0.5, 0)$);
        \coordinate (MU_BEGIN) at ($(MIDDLE_BEGIN) + (0.25, 0.5)$);
        \coordinate (SIGMA_BEGIN) at ($(MIDDLE_BEGIN) + (0.25, -0.5)$);
        \coordinate (Z_BEGIN) at ($(Z_TEXT) + (0.75, 0)$);

        \coordinate (ARROW_OUT_BEGIN) at ($(Z_TEXT) + (1.75, 0)$);
        \coordinate (ARROW_OUT_END) at ($(ARROW_OUT_BEGIN) + (0.5, 0)$);

        % Middle nodes
        \node[draw,rectangle] (SIGMA) at ($(SIGMA_BEGIN) + (0.5, 0)$) {$\sigma$};
        \node[draw,rectangle] (MU) at ($(MU_BEGIN) + (0.5, 0)$) {$\mu$};
        \node[draw,rectangle] (Z) at ($(Z_BEGIN) + (0.5, 0)$) {$z$};

        % Rectangle around all nodes
        \node[draw, rectangle, inner sep=5pt, fit=(SIGMA) (MU) (Z)] (middle_rect) {};

        % Equations below the rectangle
        \node[below=0cm of middle_rect] (eq1) {$z = \mu + \sigma \cdot \epsilon$};
        \node[below=0cm of eq1] (eq2) {$z \sim \mathcal{N}(\mu, \sigma^2)$};

        % Arrow from trapazoid to rectangle
        \draw[->] ($(E_CENTER) + (2, 0)$) -- (middle_rect);

        % Inner arrows
        \draw[->, to path={-| (\tikztotarget)}] (SIGMA) edge (Z) (MU) edge (Z);











        % Draw decoder
        \coordinate (D_CENTER) at ($(Z_TEXT) + (2.5, 0)$);
        \node at ($(D_CENTER) + (1, 1.5)$) {Decoder $p_\theta(x|z)$};

        % Arrow from the rectangle to the trapezoid
        \draw[->] (middle_rect.east) -- (D_CENTER);

        % Draw trapazoid
        \coordinate (A2) at ($(D_CENTER) + (0, 0.5)$);
        \coordinate (B2) at ($(D_CENTER) + (2, 1)$);
        \coordinate (C2) at ($(D_CENTER) + (2, -1)$);
        \coordinate (D2) at ($(D_CENTER) + (0, -0.5)$);
        \draw[fill=gray!30] (A2) -- (B2) -- (C2) -- (D2) -- cycle;
        
        % Define the coordinates for the first set of circles (2 neurons)
        \coordinate (n2_1) at ($(D_CENTER) + (0.5, 0.25)$);
        \coordinate (n2_2) at ($(D_CENTER) + (0.5, -0.25)$);
        
        % Define the coordinates for the second set of circles (3 neurons)
        \coordinate (m2_1) at ($(n2_1) + (1, 0.25)$);
        \coordinate (m2_2) at ($(n2_1) + (1, -0.25)$);
        \coordinate (m2_3) at ($(n2_1) + (1, -0.75)$);
        
        % Draw the first set of circles
        \foreach \i in {n2_1, n2_2} {
            \filldraw[fill=white] (\i) circle (6pt);
        }
        
        % Draw the second set of circles
        \foreach \i in {m2_1, m2_2, m2_3} {
            \filldraw[fill=white] (\i) circle (6pt);
        }
        
        % Draw arrows from each circle in the first set to each circle in the second set
        \foreach \i in {n2_1, n2_2} {
            \foreach \j in {m2_1, m2_2, m2_3} {
                \draw[-] (\i) -- (\j);
            }
        }


        \coordinate (D_END) at ($(D_CENTER) + (2, 0)$);

        % Draw output vector
        \coordinate (X_OUT_TEXT) at ($(m2_2) + (2, 0)$);
        \draw[->] (D_END) -- ($(X_OUT_TEXT) + (-0.25, 0)$);
        
        \draw ($(X_OUT_TEXT) + (-0.25, -1)$) rectangle ($(X_OUT_TEXT) + (0.25, 1)$) node[pos=.5] {$\hat{x}$};
        
        % \node at (X_OUT_TEXT) {$\hat{x}$};
        
    \end{tikzpicture}
    \caption{Variational Autoencoder architecture.}
    \label{figure:vae}
\end{figure}


The VAE architecture is shown in figure \ref{figure:vae}.

\subsection{Training}

The VAE optimizes the Evidence Lower Bound (ELBO) (see appendix \ref{appendix:elbo}) to ensure that the approximate posterior $q_\phi (z|x)$ is close to the true posterior $p(z|x)$ (we want to maximize it):

\begin{equation}
    \mathcal{L}(\theta, \phi; x, z) = \text{ELBO} = \mathbb{E}_{q_\phi(z|x)} \left[ \log p_\theta(x|z) \right] - D_\text{KL}(q_\phi(z|x) \| p(z))
\end{equation}

where the first term is the reconstruction loss and the second term is the KL divergence (see appendix \ref{appendix:kl_divergence}) (which measure how much the approximate posterior $q_\phi (z|x)$ diverges from the prior $p(z)$). 