
\begin{figure}[h]
    \centering
    \begin{tikzpicture}
        % Center point of encoder
        \coordinate (E_CENTER) at (1, 1);
        \coordinate (INPUT_TEXT) at (-0.25, 0.5);



        % Draw input vector
        \draw ($(INPUT_TEXT) + (-0.25, -0.5)$) rectangle ($(INPUT_TEXT) + (0.25, 1.5)$) node[pos=.5] {$x$};
        \draw[->] ($(E_CENTER) + (-1, 0)$) -- (E_CENTER);

        








        % Draw the encoder
        \node at ($(E_CENTER) + (1, 1.5)$) {Encoder $q_\phi(z|x)$};
        
        \coordinate (A) at ($(E_CENTER) + (0, -1)$);
        \coordinate (B) at ($(E_CENTER) + (2, -0.5)$);
        \coordinate (C) at ($(E_CENTER) + (2, 0.5)$);
        \coordinate (D) at ($(E_CENTER) + (0, 1)$);
        \draw[fill=gray!30] (A) -- (B) -- (C) -- (D) -- cycle;
        
        % Define the coordinates for the first set of circles (3 neurons)
        \coordinate (n1) at ($(E_CENTER) + (0.5, -0.5)$);
        \coordinate (n2) at ($(E_CENTER) + (0.5, 0)$);
        \coordinate (n3) at ($(E_CENTER) + (0.5, 0.5)$);
        
        % Define the coordinates for the second set of circles (2 neurons)
        \coordinate (m1) at ($(E_CENTER) + (1.5, 0.25)$);
        \coordinate (m2) at ($(E_CENTER) + (1.5, -0.25)$);
        
        % Draw the first set of circles
        \foreach \i in {n1, n2, n3} {
            \filldraw[fill=white] (\i) circle (6pt);
        }
        
        % Draw the second set of circles
        \foreach \i in {m1, m2} {
            \filldraw[fill=white] (\i) circle (6pt);
        }
        
        % Draw arrows from each circle in the first set to each circle in the second set
        \foreach \i in {n1, n2, n3} {
            \foreach \j in {m1, m2} {
                \draw[-] (\i) -- (\j);
            }
        }






        % Draw middle
        \coordinate (Z_TEXT) at ($(E_CENTER) + (3, 0)$);

        \coordinate (MIDDLE_BEGIN) at ($(Z_TEXT) + (-0.5, 0)$);
        \coordinate (MU_BEGIN) at ($(MIDDLE_BEGIN) + (0.25, 0.5)$);
        \coordinate (SIGMA_BEGIN) at ($(MIDDLE_BEGIN) + (0.25, -0.5)$);
        \coordinate (Z_BEGIN) at ($(Z_TEXT) + (0.75, 0)$);

        \coordinate (ARROW_OUT_BEGIN) at ($(Z_TEXT) + (1.75, 0)$);
        \coordinate (ARROW_OUT_END) at ($(ARROW_OUT_BEGIN) + (0.5, 0)$);

        % Middle nodes
        \node[draw,rectangle] (SIGMA) at ($(SIGMA_BEGIN) + (0.5, 0)$) {$\sigma$};
        \node[draw,rectangle] (MU) at ($(MU_BEGIN) + (0.5, 0)$) {$\mu$};
        \node[draw,rectangle] (Z) at ($(Z_BEGIN) + (0.5, 0)$) {$z$};

        % Rectangle around all nodes
        \node[draw, rectangle, inner sep=5pt, fit=(SIGMA) (MU) (Z)] (middle_rect) {};

        % Equations below the rectangle
        \node[below=0cm of middle_rect] (eq1) {$z = \mu + \sigma \cdot \epsilon$};
        \node[below=0cm of eq1] (eq2) {$z \sim \mathcal{N}(\mu, \sigma^2)$};

        % Arrow from trapazoid to rectangle
        \draw[->] ($(E_CENTER) + (2, 0)$) -- (middle_rect);

        % Inner arrows
        \draw[->, to path={-| (\tikztotarget)}] (SIGMA) edge (Z) (MU) edge (Z);











        % Draw decoder
        \coordinate (D_CENTER) at ($(Z_TEXT) + (2.5, 0)$);
        \node at ($(D_CENTER) + (1, 1.5)$) {Decoder $p_\theta(x|z)$};

        % Arrow from the rectangle to the trapezoid
        \draw[->] (middle_rect.east) -- (D_CENTER);

        % Draw trapazoid
        \coordinate (A2) at ($(D_CENTER) + (0, 0.5)$);
        \coordinate (B2) at ($(D_CENTER) + (2, 1)$);
        \coordinate (C2) at ($(D_CENTER) + (2, -1)$);
        \coordinate (D2) at ($(D_CENTER) + (0, -0.5)$);
        \draw[fill=gray!30] (A2) -- (B2) -- (C2) -- (D2) -- cycle;
        
        % Define the coordinates for the first set of circles (2 neurons)
        \coordinate (n2_1) at ($(D_CENTER) + (0.5, 0.25)$);
        \coordinate (n2_2) at ($(D_CENTER) + (0.5, -0.25)$);
        
        % Define the coordinates for the second set of circles (3 neurons)
        \coordinate (m2_1) at ($(n2_1) + (1, 0.25)$);
        \coordinate (m2_2) at ($(n2_1) + (1, -0.25)$);
        \coordinate (m2_3) at ($(n2_1) + (1, -0.75)$);
        
        % Draw the first set of circles
        \foreach \i in {n2_1, n2_2} {
            \filldraw[fill=white] (\i) circle (6pt);
        }
        
        % Draw the second set of circles
        \foreach \i in {m2_1, m2_2, m2_3} {
            \filldraw[fill=white] (\i) circle (6pt);
        }
        
        % Draw arrows from each circle in the first set to each circle in the second set
        \foreach \i in {n2_1, n2_2} {
            \foreach \j in {m2_1, m2_2, m2_3} {
                \draw[-] (\i) -- (\j);
            }
        }


        \coordinate (D_END) at ($(D_CENTER) + (2, 0)$);

        % Draw output vector
        \coordinate (X_OUT_TEXT) at ($(m2_2) + (2, 0)$);
        \draw[->] (D_END) -- ($(X_OUT_TEXT) + (-0.25, 0)$);
        
        \draw ($(X_OUT_TEXT) + (-0.25, -1)$) rectangle ($(X_OUT_TEXT) + (0.25, 1)$) node[pos=.5] {$\hat{x}$};
        
        % \node at (X_OUT_TEXT) {$\hat{x}$};
        
    \end{tikzpicture}
    \caption{Variational Autoencoder architecture.}
    \label{figure:vae}
\end{figure}
